\documentclass{amsart}

\begin{document}
    Let $(x_1, \ldots, x_n)$ be a multiple of $(y_1, \ldots, y_n)$.  We can
    then write the matrix
    \begin{equation}
        \notag
        \begin{bmatrix}
            x_1 && \hdots && x_n \\
            y_1 && \hdots && y_n
        \end{bmatrix} = 
        \begin{bmatrix}
            x_1 && \hdots && x_n \\
            \lambda x_1 && \hdots && \lambda x_n \\
        \end{bmatrix}
    \end{equation}
    for some scalar $\lambda$. However, we can now subtract a multiple
    $\lambda$ of the first row from the second row and achieve
    \begin{equation}
        \notag
        \begin{bmatrix}
            x_1 && \hdots && x_n \\
            0 && \hdots && 0
        \end{bmatrix}.
    \end{equation} If we now let $x_i$ be the smallest non-zero component for
    $1 \leq i \leq n$ and divide the first row by $x_i$ we achieve the
    following matrix
    \begin{equation}
        \notag
        \begin{bmatrix}
            0 && \hdots && 1 && \frac{x_i+1}{x_i} && \hdots && \frac{x_n}{x_i} \\
            0 && \hdots && 0 && 0 && \hdots && 0
        \end{bmatrix}.
    \end{equation}
    This matrix has exactly one pivot-column if there is at least one $i$ for
    which $x_i \neq 0$ and exactly zero pivot columns if $x_i = 0$ for all $i$.
    Consequently, the rank of this matrix is either 1 or 0, and more
    specifically, less than or equal to 1.
\end{document}
