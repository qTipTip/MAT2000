\documentclass[a4paper]{article}

\usepackage[]{amsmath} 
\usepackage[]{amsthm} 
\usepackage[]{amssymb} 

\usepackage[demo]{graphicx} 
\usepackage[]{caption} 
\usepackage[]{subcaption} 

\usepackage[]{cleveref} 

\newcommand{\C}{\ensuremath{\mathbb{C}}}

\title{Visualizing Algebraic Surfaces}
\author{Ivar Haugal{\o}kken Stangeby}
\date{\today}

\begin{document}
    
    \maketitle
    \tableofcontents

    \section{An Informal Introduction}
    \label{sec:an_informal_introduction}
    
    In this section we briefly look at the natural construction of the
    mathematical objects we are interested in studying in the rest of this
    paper.

    \subsection{Real Algebraic Curves}
    \label{sub:real_algebraic_curves}
    
    From elementary mathematics one learns about real valued functions, $f(x)$,
    and how to graph these functions by setting $y = f(x)$ and plotting points
    in the $(x, y)$-plane. Now, the \emph{graph of a function} is something of
    a peculiarity, because it comes with some restrictions. Not all curves in
    the $(x, y)$-plane correspond to functions. The method for verifying
    whether a certain ''graph'' corresponds to a function or not typically
    taught in school is the \emph{vertical line test}. 

    Having an equation $y = f(x)$ we can form what we call the \emph{equation
    of a curve at zero}. We define a new function in two variables
    \begin{equation}
        \label{eq:equation_at_zero}
        g(x, y) = y - f(x) = 0.
    \end{equation}
    If the function $f(x)$ is a polynomial in the variable $x$ with certain
    coefficients we call the function graph of $f(x)$ \emph{algebraic}.

    Generally speaking, we call a curve defined by \cref{eq:equation_at_zero}
    an \emph{algebraic curve} if the function $g(x, y)$ is a polynomial in two
    variables, $x$ and $y$. Mathematically, this can be expressed as
    \begin{equation}
        \label{eq:algebraic_curve}
        g(x, y) = \sum^{}_{i, j} a_{i_j}x^{i}y^{i}.
    \end{equation}
    
    However, if this distinction between a graph and a curve is to be justified
    there has to be some curves that are not graphs. One of the first examples
    one encounter of a curve not corresponding to a function is what you get
    when you set $y^2 = x^3$. This curve does not pass the vertical-line-test
    and is therefore something different from a graph. Similarly, the equation
    $y^2 = x^3 - x^2$ defines a curve that again is not a graph. These are
    shown in \cref{fig:curves_not_graphs}. These curves exhibit \emph{singular
    points} at $(0, 0)$. 
    
    \begin{figure}[h]
        \centering
        \begin{subfigure}{0.5\textwidth}
            \centering
            \includegraphics[width=0.35\linewidth]{foo}
            \caption{$g(x, y) = y^2 - x^3 = 0$}
        \end{subfigure}%
        \begin{subfigure}{0.5\textwidth}
            \centering
            \includegraphics[width=0.35\linewidth]{bar}
            \caption{$g(x, y) = y^2 - x^3 + x^2 = 0$}
        \end{subfigure} 
        \caption{Examples of curves that are \emph{not} graphs given by their
        functions $g(x, y)$}.}
        \label{fig:curves_not_graphs}
    \end{figure}
    
    With what we have so far, we can move up a dimension. Instead of
    considering curves in the $(x, y)$-plane we can look at surfaces in the
    $(x, y, z)$-space. We, as we did with $g(x, y)$ above, define a function
    $h(x, y, z) = z - g(x, y)$. The surfaces are given by equations on the form
    $h(x, y, z) = 0$. These surfaces are called algebraic if $h(x, y, z)$ is a
    polynomial in the variables $x, y, z$. Again, mathematically, this is
    expressed as
    \begin{equation}
        \label{eq:algebraic_surfaces}
        h(x, y, z) = \sum^{}_{i, j, k} a_{ijk}x^iy^jz^k.
    \end{equation}

    \subsection{Going Complex}
    \label{sub:going_complex}
    
    The surfaces considered so far are over the real numbers, i.e, the
    coefficients $a_{ijk}$ are real numbers. We can instead work over the
    complex numbers where the surfaces are given by an equivalent equation, but
    where the coefficients now are complex numbers\footnote{We will come back
    to why we make this transition when we talk about sets being algebraically
    closed or not.}:
    \begin{equation}
        \left\{ (x, y, z) \in \C \mid h(x, y, z) = 0 \right\}.
    \end{equation}
    Again, doing the same trick (one-trick ponies) we can now define $w = h(x,
    y, z)$ and look at the equations $w - h(x, y, z) = 0$. Now, we are stuck
    with an equation in four variables. With only three degrees of freedom to
    work with when visualizing these objects we encounter an obstacle. How do
    we know what these surfaces look like? This brings us to the world of
    \emph{projective geometry}.

    \section{Projective Geometry}
    \label{sec:projective_geometry}
    
        


\end{document}
